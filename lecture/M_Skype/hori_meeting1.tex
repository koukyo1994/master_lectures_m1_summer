\documentclass[a4paper,10pt]{jsarticle}
\usepackage[utf8]{inputenc}
\usepackage[dvipdfmx]{graphicx}
%opening
\title{修士研究のテーマについて}
\author{M1, 荒居 秀尚}

\begin{document}

\maketitle

\section{研究の方針}
先日、このミーティングを設定していただいたときには、テーマの方針について悩んでいたのですが、
月曜日に矢入先生と話す機会をいただきまして、そこでテーマ選びについての不安事項などは払拭出来ました。\\

今、自分の考えているテーマについては自分の中で幾つか方針がある(ことに最近気がつきました)のではじめに書いておこうと思います。\\

\begin{itemize}
 \item 人気が加熱している領域には突っ込みたくない
 \item 実用上広い範囲で役に立つものを研究できれば嬉しい
 \item できれば自分で考えたアイデアで勝負がしたい
\end{itemize}


\section{現在考えている研究テーマ}

昨年の5月あたりからやってみたいと考えていたテーマとして、ドローンの交通規則というものがあります。また、別の方向性として
自動運転車の車間通信、路肩通信ネットワークというものも研究テーマとして面白いのではないかと考え3月ごろから調べてきました。\\

先に、自動運転車の車間通信・路肩通信ネットワークについてなのですが、これに関しては自動運転車が前方の車両や道路に埋め込まれた
センサーから情報を入手することができるようになれば、もっと自動運転車の運転に有利になるのではないかと考えたのですが、同様のアイデアとしては
VANET(Vehicular Ad Hoc Network)\cite{vanet}や、Content-centric networking\cite{content_centric}といった分野で2010年ごろから
研究がされているようです。\\

ドローンの交通規則という方に関しては、最近開発が進んでいる有人ドローンというものを先に紹介させていただきます。
有人ドローンは図\ref{fig:ehang}のように、人が乗ることを想定した少人数向けの航空機で近年いくつかの企業が開発を進めています。

\begin{figure}[htbp]
 \begin{center}
  \includegraphics[clip, scale=0.80]{ehang.jpeg}
  \caption{有人ドローンの例(Ehang 184)}
  \label{fig:ehang}
 \end{center}
\end{figure}

個人的な意見なので論理的な根拠はないのですが、僕はこれが将来的には車に置き換わって普及するのではないかと考えています(というより普及してほしい、に近いかもしれません)。
その際には必ず有人ドローンのための交通規則というものが出てくるのではないかと思うのですが、それを予め研究してしまうことができたらいいな、と考え最近は興味を持っています。\\

また、個人的な考えなのですがこのような乗り物はデフォルトで自動運転のものとして現れると考えているので(というのも既に出ているものはすべて何らかの形で自動運転のシステムを使っています)、
自動運転のシステムが乗ったドローンが先述の車間通信や路肩通信を使うことで交通規則をシステムに組み込みのものとして社会実装することができるのではないかというのがコンセプトです。\\

先日、西成先生への連絡をしてよいか、ご相談させていただいたのは、このようなことを考えていた折に、柳澤研の同期と話していて西成先生が2018年に研究したいことの一つとして
ドローンの交通規則というものをあげていた、と聞いたという経緯があります。\\

この話について矢入先生に相談させていただいたところ、このテーマで進めるのはいいと思うが切り口が難しいね、とのフィードバックをいただきました。また、西成先生に相談するとしたら
もう少し具体的に切り口を考えてある程度研究を進めてから(シミュレーションをするなど)の方がいいとのことでしたので、それに従いひとまず切り口を考えてみようと思っています。\\

切り口の案として矢入先生からご提示いただいたものとしては以下のようなものがあります。

\begin{itemize}
 \item 分散制御か中央管理型の制御かの対比
 \item (西成・柳澤研でおこなっているように)ある程度モデル化をしてそのダイナミクスを解析
 \item 二次元のセル・オートマトンを三次元に拡張するような形で論ずる
 \item 法整備に寄った研究(矢入先生はあまり面白くならないかもしれないけどとおっしゃっていました。)
 \item 外乱環境下でも飛べるのかといった議論
 \item そもそもどのような状況でなら実行可能性があるのかといったフィージビリティスタディ
\end{itemize}

正直なところ、まだ世に出ていないものを研究することはいろいろな面でリスキーなことかと思いますのでテーマとしてこの方向性で検討することにまだ不安はあるのですが、2年あれば
ある程度のことはできるのではないかとも期待しています。

\bibliographystyle{junsrt}
\bibliography{researchinterest}

\end{document}
